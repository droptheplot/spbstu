\documentclass[a4paper,12pt,russian]{report}
\usepackage[utf8]{inputenc}
\usepackage[T2A]{fontenc}
\usepackage{amsmath}
\usepackage{listings}
\usepackage{xcolor}
\usepackage{tabularx}
\usepackage{geometry}
\usepackage{tikz}
\usepackage{titlesec}
\usepackage[font=small,labelfont=bf]{caption}
\usepackage{indentfirst}

\usetikzlibrary{shapes,arrows}
\tikzstyle{line} = [draw, -latex']
\tikzstyle{decision} = [
  diamond,
  aspect=3,
  draw,
  fill=yellow!20,
  % text width=5em,
  text badly centered,
  node distance=2cm,
  % inner sep=0pt
]
\tikzstyle{process} = [
  rectangle,
  draw,
  fill=blue!20,
  node distance=2cm,
  % text width=10em,
  text centered,
  minimum height=3em
]
\tikzstyle{terminal} = [
  rectangle,
  draw,
  fill=red!20,
  node distance=2cm,
  text width=5em,
  text centered,
  minimum height=3em,
  rounded corners
]
\tikzstyle{io} = [
  trapezium,
  trapezium left angle=120,
  trapezium right angle=60,
  draw,
  fill=green!20,
  node distance=2cm,
  % text width=5em,
  text centered,
  minimum height=3em
]

\lstdefinelanguage{clojure} {
  morekeywords={defn, as->, ->>, try, catch, ns},
  morestring=[b]",
  morestring=[b]',
}

\newcommand{\insertInstitute}{
  Институт компьютерных наук и технологий\linebreak
  Высшая школа киберфизических систем и управления
}
\newcommand{\insertTitle}{ОТЧЕТ\par по дисциплине «Информатика»\par \textbf{Вычисление формулы для диапазона чисел}}
\newcommand{\insertAuthor}{С. А. Новиков}
\newcommand{\insertAuthorPosition}{студент гр 13532/1}
\newcommand{\insertVerifier}{C. В. Хлопин}
\newcommand{\insertVerifierPosition}{доцент, к.т.н.}

\newcommand{\sectionbreak}{\clearpage}

\sloppy

\linespread{1.3}
\definecolor{lightgray}{gray}{0.95}
\renewcommand{\contentsname}{Содержание}
\renewcommand{\thesection}{\arabic{section}}
\newgeometry{left=3cm,right=2cm,top=2cm,bottom=2cm}
\setlength{\parindent}{1.25cm}
\lstset{
  backgroundcolor=\color{lightgray},
}


\begin{document}

\pagenumbering{gobble}
\begin{center}
  Министерство науки и высшего образования РФ\linebreak
  Санкт-Петербургский политехнический университет\linebreak
  Петра Великого\linebreak
  \insertInstitute\linebreak
\end{center}
\vspace{1.5cm}
\begin{tabularx}{\textwidth}{Xr}
  УДК $\rule{4cm}{0.15mm}$ & УТВЕРЖДАЮ \\
                           & $\rule{5cm}{0.15mm}$ \\
                           & $\rule{5cm}{0.15mm}$ \\
                           & $\rule{5cm}{0.15mm}$ \\
                           & «$\rule{0.8cm}{0.15mm}$» $\rule{2cm}{0.15mm}$ $\rule{1.1cm}{0.15mm}$ г. \\
\end{tabularx}
\vspace{1.5cm}
\begin{center}
  \insertTitle\par
\end{center}
\vspace{1.5cm}
\begin{tabularx}{1\textwidth}{Xll}
  \textbf{Выполнил:}    & & \\
  \insertAuthorPosition & $\rule{3.5cm}{0.15mm}$ & \insertAuthor \\
                        & подпись, дата & \\
  \textbf{Проверил:}      & & \\
  \insertVerifierPosition & $\rule{3.5cm}{0.15mm}$ & \insertVerifier \\
                          & подпись, дата & \\
\end{tabularx}
\vfill
\begin{center}
  Санкт-Петербург $\rule{1.1cm}{0.15mm}$ г.
\end{center}


\newpage
\pagenumbering{arabic}
\setcounter{page}{2}

\tableofcontents

\section{Реферат}

Отчет 5 с.
ПРОГРАММИРОВАНИЕ, ЯЗЫКИ ПРОГРАММИРОВАНИЯ
Объектом исследования является написание программы для вычисления выражения по формуле.
Цель работы — создать программу на любом языке программирования для вычисления значения Х и вывода сообщения об ошибке при невозможности вычисления значения.

\section{Блок-схема}

\begin{center}
\begin{tikzpicture}[node distance = 2cm, auto]
  \node [terminal] (start) {Начало};
  \node [io, below of=start] (ask-start-x) {Запросить начальное Х};
  \node [io, below of=ask-start-x] (read-start-x) {Прочитать начальное Х};
  \node [io, below of=read-start-x] (ask-end-x) {Запросить конечное Х};
  \node [io, below of=ask-end-x] (read-end-x) {Прочитать конечное Х};
  \node [io, below of=read-end-x] (ask-delta-x) {Запросить дельту Х};
  \node [io, below of=ask-delta-x] (read-delta-x) {Прочитать дельту Х};
  \node [decision, below of=read-delta-x] (loop) {Количество значений};
  \node [process, below of=loop] (mincoscot) {Вычисление формулы для каждого значения};
  \node [io, below of=mincoscot] (print) {Вывод таблицы на экран};
  \node [terminal, below of=print] (end) {Конец};

  \path [line] (start) -- (ask-start-x);
  \path [line] (ask-start-x) -- (read-start-x);
  \path [line] (read-start-x) -- (ask-end-x);
  \path [line] (ask-end-x) -- (read-end-x);
  \path [line] (read-end-x) -- (ask-delta-x);
  \path [line] (ask-delta-x) -- (read-delta-x);
  \path [line] (read-delta-x) -- (loop) ;
  \path [line] (loop) -- (mincoscot);
  \draw [->] (mincoscot.west) |- (loop.west);
  \draw [->] (loop.east) -| (print.east);
  \path [line] (print) -- (end);
\end{tikzpicture}
\end{center}

\section{Вывод}

В ходе выполнения данного задания было обнаружено что в случае деления на ноль или взятия логарифма из отрицательного числа вычисление невозможно.

\end{document}
