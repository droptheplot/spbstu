\documentclass[a4paper,12pt,russian]{report}
\usepackage[utf8]{inputenc}
\usepackage[T2A]{fontenc}
\usepackage{amsmath}
\usepackage{listings}
\usepackage{xcolor}
\usepackage{tabularx}
\usepackage{geometry}

\newcommand{\insertInstitute}{
  Институт компьютерных наук и технологий\linebreak
  Высшая школа «Киберфизических систем и управления»
}
\newcommand{\insertTitle}{Название работы}
\newcommand{\insertAuthor}{И. О. Выполняющий}
\newcommand{\insertVerifier}{И. О. Проверящий}
\newcommand{\insertVerifierPosition}{Главный проверяющий}

\linespread{1.3}
\definecolor{lightgray}{gray}{0.95}
\newgeometry{left=3cm,right=2cm,top=2cm,bottom=2cm}
\setlength{\parindent}{1.25cm}
\lstset{
  keywordstyle = \fsize\color[HTML]{71295b},
  commentstyle = \fsize\color[HTML]{839496},
  stringstyle = \fsize\color[HTML]{2429c9},
}
\renewcommand{\thesection}{\arabic{section}}
\renewcommand{\contentsname}{Содержание}
\renewcommand{\figurename}{Рисунок}
\renewcommand{\tablename}{Таблица}


\begin{document}
\pagenumbering{gobble}
\begin{center}
  Министерство образования и науки РФ\linebreak
  Санкт-Петербургский политехнический университет\linebreak
  Петра Великого\linebreak
  \insertInstitute\linebreak
\end{center}
\vspace{2cm}
\begin{tabularx}{\textwidth}{Xr}
  УДК $\rule{4cm}{0.15mm}$ & УТВЕРЖДАЮ \\
                           & $\rule{5cm}{0.15mm}$ \\
                           & $\rule{5cm}{0.15mm}$ \\
                           & $\rule{5cm}{0.15mm}$ \\
                           & «$\rule{0.8cm}{0.15mm}$» $\rule{2cm}{0.15mm}$ $\rule{1.1cm}{0.15mm}$ г. \\
\end{tabularx}
\vspace{2cm}
\begin{center}
  ОТЧЕТ\par
  О НАУЧНО-ИССЛЕДОВАТЕЛЬСКОЙ РАБОТЕ\par
  \textbf{\insertTitle}\par
\end{center}
\vspace{2cm}
\begin{tabularx}{1\textwidth}{Xll}
  \textbf{Выполнил:}                 & & \\
  Студент гр. $\rule{2cm}{0.15mm}$ & $\rule{3.5cm}{0.15mm}$ & \insertAuthor \\
                                     & подпись, дата & \\
  \textbf{Проверил:}      & & \\
  \insertVerifierPosition & $\rule{3.5cm}{0.15mm}$ & \insertVerifier \\
                          & подпись, дата & \\
\end{tabularx}
\vfill
\begin{center}
  Санкт-Петербург $\rule{1.1cm}{0.15mm}$ г.
\end{center}


\newpage
\tableofcontents
\clearpage
\pagenumbering{arabic}

\section{Задание}

\subsection{Описание}
На языке программирования создать программу, которая будет просить у пользователя ввести начальные и конечные значения для диапазона расчета X, шаг изменения переменной deltaX. Программа должна вывести на экран таблицу (которая корректно выводит значения для разного набора исходных данных - столбцы «не едут») с номером строки, значению X, значению полученного выражения. В случае невозможности вычисления выражения для конкретного случая X, num (деление на ноль, логарифм из отрицательного числа, значение синуса или консинуса равно 0), в строке таблицы необходимо вывести сообщение об ошибке.

\subsection{Формула}
\begin{equation*}
  f(x) = \min\left(\ln\left(1-\frac{61}{\cos(x)}\right), \frac{\cot(x)}{61}\right)
\end{equation*}
\clearpage

\section{Решение}

\subsection{Описание}
Решение выполнено на языке программирования общего назначения Clojure, современном диалектe Lisp, работающем на платформе JVM.

\subsection{Запуск}
Для запуска решения необходимо запустить \textbf{clj}, указав название нужного пространства имен с функцией \textbf{main}.
\begin{lstlisting}[language=bash]
clj -m mincosctg
\end{lstlisting}

\subsection{Пример}
После запуска решения программа попросит ввести начальное значение X, конечное значение X и шаг изменения X. После чего выведет на экран таблицу с результатами вычисления формулы для каждого значения X или \textbf{Error}, если такое вычисление невозможно.
\begin{lstlisting}[language=bash]
Enter start:
0
Enter end:
20
Enter delta:
2

| :x |      :min |
|----+-----------|
|  0 |     Error |
|  2 | -0.007503 |
|  4 |  0.014159 |
|  6 |     Error |
|  8 | -0.002411 |
| 10 |  0.025284 |
| 12 |     Error |
| 14 |     Error |
| 16 |  0.054530 |
| 18 |     Error |
\end{lstlisting}

\end{document}
