\documentclass[a4paper,14pt,russian]{report}
\usepackage[utf8]{inputenc}
\usepackage[T2A]{fontenc}
\usepackage{extsizes}
\usepackage{amsmath}
\usepackage{listings}
\usepackage{xcolor}
\usepackage{tabularx}
\usepackage{geometry}
\usepackage{titlesec}
\usepackage{graphicx}
\usepackage[font=small,labelfont=bf]{caption}
\usepackage{indentfirst}

\newcommand{\insertInstitute}{
  Институт компьютерных наук и технологий\linebreak
  Кафедра «Информационные системы и технологии»
}
\newcommand{\insertTitle}{МИКРОРЕФЕРАТ\par \textbf{Роль слухов в бизнесе}}
\newcommand{\insertAuthor}{С. А. Новиков}
\newcommand{\insertAuthorPosition}{студент гр 13532/1}
\newcommand{\insertVerifier}{Н. В. Анисина}
\newcommand{\insertVerifierPosition}{доцент, канд. пед. наук}

\newcommand{\sectionbreak}{\clearpage}
\newcommand{\subsectionbreak}{\clearpage}

\graphicspath{ {./images/} }

\sloppy

\linespread{1.3}
\definecolor{lightgray}{gray}{0.95}
\renewcommand{\contentsname}{Содержание}
\renewcommand{\thesection}{\arabic{section}}
\newgeometry{left=3cm,right=2cm,top=2cm,bottom=2cm}
\setlength{\parindent}{1.25cm}
\lstset{
  backgroundcolor=\color{lightgray},
}


\begin{document}

\pagenumbering{gobble}
\begin{center}
  Министерство науки и высшего образования РФ\linebreak
  Санкт-Петербургский политехнический университет\linebreak
  Петра Великого\linebreak
  \insertInstitute\linebreak
\end{center}
\vspace{1.5cm}
\begin{tabularx}{\textwidth}{Xr}
  УДК $\rule{4cm}{0.15mm}$ & УТВЕРЖДАЮ \\
                           & $\rule{5cm}{0.15mm}$ \\
                           & $\rule{5cm}{0.15mm}$ \\
                           & $\rule{5cm}{0.15mm}$ \\
                           & «$\rule{0.8cm}{0.15mm}$» $\rule{2cm}{0.15mm}$ $\rule{1.1cm}{0.15mm}$ г. \\
\end{tabularx}
\vspace{1.5cm}
\begin{center}
  \insertTitle\par
\end{center}
\vspace{1.5cm}
\begin{tabularx}{1\textwidth}{Xll}
  \textbf{Выполнил:}    & & \\
  \insertAuthorPosition & $\rule{3.5cm}{0.15mm}$ & \insertAuthor \\
                        & подпись, дата & \\
  \textbf{Проверил:}      & & \\
  \insertVerifierPosition & $\rule{3.5cm}{0.15mm}$ & \insertVerifier \\
                          & подпись, дата & \\
\end{tabularx}
\vfill
\begin{center}
  Санкт-Петербург $\rule{1.1cm}{0.15mm}$ г.
\end{center}


\newpage
\pagenumbering{arabic}
\setcounter{page}{2}

\section*{Реферат}

\noindent Отчет 9 с., 5 источников. \\
БИЗНЕС СЛУХИ КОНКУРЕНЦИЯ СМИ \\
Объектом исследования являются слухи в бизнесе. \\
Цель работы - изучение влияния слухов на бизнес. \\

\tableofcontents

\section{Введение}

В современном мире принято считать, что слухи возникли в тот же момент времени, что и устный вид коммуникации. В течение долгого промежутка времени передача информации проходила из уст в уста при помощи песен, баллад, рассказов из мифологии, а кроме того – что проще и наиболее понятнее – слухов и сплетен. Но стоит отметить что, не обращая внимания на проходящий во всем мире процесс глобализации, слухи продолжают занимать существенное место в жизни общества. Слухи подразумевают под собой желаемую информацию. Даже факты в негативном ключе в них могут быть приняты на веру. Хотя, на первый взгляд, казалось бы, человек обязан уметь отделять рациональное от иррационального, то что можно доказать, от того что нельзя, истинное от ложного или недостаточно подтвержденного. Но такой является человеческая психология, особенности восприятия разной информации. Слухи зачастую более притягательны, чем полностью правдивые факты.

Процесс распространения слухов считается наиболее эффективным среди других инструментов ходе конкурентной борьбы в бизнесе. Профессионально запущенный слух может способствовать как к краху компании, так и к большому расцвету. Вот почему следует серьезно рассматривать роль слухов в жизни общества, а в особой степени – в бизнесе.

\section{Роль слухов в бизнесе}

Слух представляет собой вид специфической межличностной коммуникации неформального вида, в ходе которой сюжет, до определенной степени отображающий некоторые вымышленные или реальные события, становится достоянием широкой смешанной людской аудитории1. Иными словами, слух - это сообщение, правдивость которого на момент распространения не точно установлена, о наиболее важном для аудитории событии, которое передается от одного человека другому устно, а также представляет собой один из самых эффективным способов оказания влияния на общественное сознание.

В большей степени важное свойство любого слуха – его самостоятельная трансляция. Самостоятельная трансляция сообщения такова, что её сложно удержать в себе. Человек в любом случае будет стараться передать его дальше, а передав, будет испытывать облегчение психологического вида.
Следует предположить ряд объяснений данной особенности. В первую очередь, слух несет в себе информацию, которая может умалчивать в средствах массовой информации. Исходя из этого, став доступной, она без особого труда может передаваться от человека к человеку. Наряду с этим, верно и обратное, слух никогда не будет дублировать, или не будет цитировать информацию «для всех», которая содержится в СМИ.

Во вторую очередь в наиболее широком смысле, слух может представлять собой проявление коллективного бессознательного. Слухи заключают в себе ответы на тревожные ожидания коллектива, которые хранятся в душе у каждого отдельного человека; могут являться ответом на некое желание социума, например, нести в себе информацию о возможном визите чиновника высокого ранга с целью «ревизии» порядка на местах; может нести сведения об популярных или известных людях. Темы такого рода всегда вызывают большой интерес у всей аудитории.

В третью очередь, слух представляет собой общение толпы. Своеобразный ответ на желание всего общества, в нем практически полностью исключен интерес индивидуального человека.
Слухи существуют как очень важный фактор формирования мнения социума, имиджа отдельной личности или организации. На фоне распространенных слухов способна зародиться паника, они очень часто существенно компрометируют органы управления государством, могут являться причиной политической дестабилизации со стороны общества.Исходя из вышесказанного, слух – это сильный коммуникативный инструмент, который может заставить задуматься о его роли и применении очень серьезно.

В качестве главных направлений применения слухов в современный момент времени, следует выделить:

\begin{enumerate}
  \item Создание особого имиджа личности, организации, фирмы, а также процесс манипулирования общественным мнением.
  \item Намеренное введение конкурента в заблуждение.
  \item Привлечение внимания к конкретному событию или личности. В таком случае слухи активно применяются в сфере шоу-бизнеса. Наиболее часто это бывают слухи о каких-либо недоразумениях, скандалах, подробностях жизни интимного характера.
  \item Реклама различных товаров или услуг.
  \item Провокация населения на совершение каких-либо действий, являющихся выгодными для одной из сторон, конфликтующих между собой. Речь в данном случае может идти о массовых беспорядках, забастовках, спросе на продукты в виде повышенного ажиотажа и т.п.
  \item Подготовка общества или персонала организации к решениям непопулярного характера.
\end{enumerate}

Но стоит учитывать, что у слухов есть и другая сторона. Без учета особенностей слухов работа с ними не всегда может приводить к желаемым результатам. Находясь в массах, слухи очень часто могут поддаваться серьезным изменениям, что без труда приводит к полному изменению смысла начального варианта слуха.

Слух представляет собой инструмент, который в неумелых руках способен быть опасным даже для своего непосредственного создателя. Именно поэтому, создавая тот или иной слух, нужно быть очень осмотрительным, в большей степени, если речь идет о сфере бизнеса. В большей мере наглядным примером оказания влияния слухов на бизнес представляется фондовый рынок. Котировки акций могут меняться в зависимости от характера информации которая поступает из вне (негативная или позитивная).

Также очень часто можно наблюдать и противоположную ситуацию: две компании, конкурирующие между собой, которые претендует на товар или услугу одного вида (например, аренда площадей) начинает распускать друг про друга слухи порочащего вида о недобросовестности, ненадежности или недостаточной платежеспособности. В итоге это может привести к тому, что до арендодателя могут дойти нелестные отзывы об одной из данных компаний, и он предпочтет ее другой, опираясь только на слух.

Каким образом можно бороться со слухами? Ряд бизнесменов, которые столкнулись с негативными слухами в адрес своих компаний, рекомендуют запускать свои собственные слухи. Другие – не предпринимать ничего, не оправдываться, вести по максимуму открытую и прозрачную политику по информированию и сохранять хладнокровие. Третьи утверждают, что бороться со слухами нужно, только позволив социуму убедиться в их полной несостоятельности.

\section{Заключение}

Слухи являются важным фактором формирования общественного мнения, имиджа личности, организации. Они крайне важны в определенных сферах экономики, например в банковском бизнесе, когда одного негативного слуха порой бывает достаточно, чтобы вызвать серьезный отток вкладов. На фоне слухов может порождаться паника, они зачастую серьёзно компрометируют органы государственного управления, способствуют политической дестабилизации общества. Таким образом, слух - это довольно мощный коммуникативный инструмент, что заставляет задуматься о его использовании более серьезно.

По мере развития человеческого общества и средств массовых коммуникаций, роль слухов увеличивалась. Сейчас сложно представить жизнь без этого явления, ведь слухи стали частью повседневной коммуникации. Более того, слухи применяются и в политических играх, и в бизнесе, и в других сферах жизни общества. На первый взгляд слух кажется очень простым в применении. Но слухи могут не только приносить пользу, но и нанести непоправимый ущерб. Поэтому очень важно знать все аспекты и сложности применения слухов, а также уметь предотвратить их появление или пресечь уже появившийся слух.

\section{Список литературы}

\begin{enumerate}
  \item Беззубцев С. Слухи, которые работают на Вас. Секреты профессионального использования // Москва, Питер, - 2003. 239 с.
  \item Ганопольская, Е.В., Волошинова, Т.Ю., Анисина Н.В. Русский язык и культура речи. Семнадцать практических занятий // Москва, Питер, - 2012. – 336 с.
  \item Караяни, А.Г. Слухи как средство информационно-психологического противодействия // Психологический журнал, Том 24, №6, 2003. - 162 с.
  \item Мамонтов А. Слухи и современное общество // Журнал Сообщение, 2002. - 192 с.
  \item Ширяев О. Слухи встали на службу бизнесу // Советник, 2002. - 237 с.
\end{enumerate}

\end{document}
